\documentclass[a4paper, 11pt]{exam}
\usepackage[T1]{fontenc}
\usepackage{titling}
\usepackage{url}
\usepackage{amsmath,amsthm,amssymb}
\usepackage{graphicx}
\usepackage{graphics}
\usepackage{listings}
\usepackage{color} % Required for custom colors
\usepackage[dvipsnames]{xcolor}
\usepackage{tabularx}
\usepackage{ragged2e}
\usepackage{courier}
\usepackage{textcomp}
\usepackage{circuitikz}
\usepackage{tikz}
\usepackage{enumitem}
\usepackage{karnaugh-map}
\usepackage{bytefield}
\usepackage{mathrsfs}
\usepackage{cancel}
\usepackage[linesnumbered,ruled,vlined]{algorithm2e}
\usepackage{hyperref}
\usepackage{amsmath}
\usepackage{my_styles}  % This includes my custom style definitions for code

\newcommand{\invlaplace}[1]{%
\mathscr{L}^{-1}\left\{#1\right\}
}
\newcommand{\laplace}[1]{%
\mathscr{L}\left\{#1\right\}
}
\newcommand{\fourier}[1]{%
\mathscr{F}\left\{#1\right\}
}

\newcommand{\ztransform}[1]{%
\mathscr{Z}\left\{#1\right\}
}

\newcommand{\subtitle}[1]{%
  \posttitle{%
    \par\end{center}
    \begin{center}\large#1\end{center}
    }%
}

\usepackage{environ}

\NewEnviron{eqnsection}[2]{%
  \newcommand{\myvspace}{#1}%
  \vspace{\myvspace}%
  \begin{align*}
  \intertext{#2}
  \BODY
  \end{align*}%
  \vspace{\myvspace}%
}



\newlist{myenumerate}{enumerate}{2}
\setlist[myenumerate,1]{label=\roman*)}
\setlist[myenumerate,2]{label=\alph*)}



\newcommand\tab[1][1cm]{\hspace*{#1}}

\renewcommand{\labelenumi}{\alph{enumi})}

\title{Team Project: Octave Band Filtering}
\subtitle{ECE 6530: Digital Signal Processing \\
\today\\}
\author{ Tyler Bytendorp \and Miguel Gomez \and Chase Griswold \and Benjamin Hayes \and\\
University of Utah Electrical and Computer Engineering}
\date{Due Date: Dec 5, 2023}

\begin{document}
\maketitle
\noindent

\section*{5.1 Octave Bands}
This bit was taken care of in Matlab and this is the function to set up the filter bands table. The results are below and a copy of the code is included in appendix A:
\begin{table}[!ht]
\centering
\begin{tabular}{|l|c|c|c|c|c|c|c|}
\hline
val [units]   & $O_0$ & $O_1$ & $O_2$ & $O_3$ & $O_4$ & $O_5$ & $O_6$ \\
\hline
Lower (Hz)   & 65.406   & 130.81   & 261.63   & 523.25   & 1046.5   & 2093     & 4186    \\
\hline
Lower (Rad)  & 0.05137  & 0.10274  & 0.20548  & 0.41096  & 0.82192  & 1.6438   & 3.2877  \\
\hline
Upper (Hz)   & 123.47   & 246.94   & 493.88   & 987.77   & 1975.5   & 3951.1   & 7902.1  \\
\hline
Upper (Rad)  & 0.096974 & 0.19395  & 0.3879   & 0.77579  & 1.5516   & 3.1032   & 6.2063  \\
\hline
Center (Hz)  & 94.439   & 188.88   & 377.75   & 755.51   & 1511     & 3022     & 6044.1  \\
\hline
Center (Rad) & 0.074172 & 0.14834  & 0.29669  & 0.59338  & 1.1868   & 2.3735   & 4.747   \\
\hline
\end{tabular}
\caption{Frequency Ranges for Octaves 0 to 6 starting with $C_2$}
\end{table}
Since we have a limit on the bands we can recognize that arises from the use of the sampling freq of $8kHz$, we can only obtain unique detection for the set of Octaves whose frequencies are below the Nyquist rate of $\frac{fs}{2}$ or $4kHz$. Or, $O_5$ in the table above.

\section*{5.2 Octave Filter Bank}
The band pass filter bank was put together in python and has the added benefit of being more portable than the use of Matlab for this application. These are the results from the creation of the x signal needed to pass on to the filter bank. Defining the expression below:
\begin{center}
  \[
    x(t) =
    \begin{cases}
      a & t , \\
      b & t , \\
      c & t
    \end{cases}
    \]
  \end{center}
  \newpage
  and now the code to do this along with the plot showing the data is what we expect:
  \newpage
\section*{Appendix A}
\begin{lstlisting}[language=Matlab]
function print_octaves(n,fs)
% n should be a range relative to A4. ex: -1:1 gives A3,A4,A5
A4 = 440;
C4 = A4.*2.^(-9./12);
B4 = A4.*2.^(2./12);
Octaves = 2.^n;
Cs = C4.*Octaves;
Bs = B4.*Octaves;
n_range = 0:length(n)-1;
Centers = (Cs + Bs)./2;
cell(size(Centers));
octave_array = arrayfun(@(x) sprintf('Octave %d', x), n_range, 'UniformOutput', false);
w_Cs = Cs.*2.*pi./fs;
w_Bs = Bs.*2.*pi./fs;
w_Centers = Centers.*2.*pi./fs;
rows = {'Lower (Hz)','Lower (Rad)','Upper (Hz)','Upper (Rad)','Center (Hz)','Center (Rad)'};
% Summarize data in a table
T = array2table([Cs; w_Cs; Bs; w_Bs; Centers; w_Centers],'VariableNames',octave_array,'RowName',rows);
disp(T)
disp('Hz are not normalized.')
disp('Radians are normalized by sampling frequency.')
end 
\end{lstlisting}
\end{document}
